\chapter{Literature Review} 
\label{chp:literature}

    \textbf{Related work}\
    
    In recent years, 3D printing technologies, such as laser powder fusion bed processes, have shown great potential for revolutionizing traditional manufacturing methods and changing the way we think about creating metal objects. To improve the quality and consistency of these parts, researchers have been investigating the use of process monitoring methods.
    One such method is the detection of lack-of-fusion flaws, which can occur in powder bed fusion (PBF) additive manufacturing of metal components. Abdelrahman et al. (2017) \nocite{abdelrahman2017flaw} demonstrated a method for detecting these flaws by monitoring the fabrication of each layer before and after laser scanning with high resolution optical imaging. Scime et al. (2018)\nocite{scime2018anomaly} presented an in-situ monitoring and analysis approach for powder bed images, which has the potential to become a component of a real-time control system in an laser powder-bed fusion (L-PBF) machine. Many of these defects are related to interactions between the recoater blade, which spreads the powder, and the powder bed. To autonomously detect and classify these spreading anomalies, Scime et al. (2018)\nocite{scime2018multi} propose the use of a Convolutional Neural Network (CNN) with a modified input layer for in-situ detection and classification of defects in Laser Powder Bed Fusion Additive Manufacturing processes. The modifications to the CNN enable the algorithm to learn the appearance of anomalies and contextual information at multiple size scales, improving its flexibility and classification accuracy while mitigating human biases. Mahmoudi et al. (2019)\nocite{mahmoudi2019layerwise} proposed a method for detecting anomalies in L-PBF metal additive manufacturing process using high-speed thermal imaging. This method involves generating a process signature from thermal images, comparing it to a reference signature to identify potential regions of interest (ROIs), modeling the spatial dependence of the ROIs using a Gaussian process model, and implementing a classifier to determine if the process is in- or out-of-control. The effectiveness of the proposed method was demonstrated through a case study on a commercial L-PBF system.  Yadav et al. (2020)\nocite{yadav2020situ} conducted a review of the use of in situ monitoring systems and data analytics for the detection of process defects in the laser powder bed fusion process. This review discussed the types of defects that can be monitored, the working principles of common in situ sensing sensors, and commercially available in situ monitoring devices. It also highlighted the challenges of post-processing in situ data and automated defect detection, and noted that machine learning approaches are being explored to develop a feedback control loop system for real-time monitoring. Scime et al. (2020) \nocite{scime2020layer} introduce a novel Convolutional Neural Network architecture for pixel-wise localization (semantic segmentation) of layer-wise powder bed imaging data in metal Additive Manufacturing processes. Key advantages of the algorithm include its ability to return segmentation results at the native resolution of the imaging sensor, seamlessly transfer learned knowledge between different Additive Manufacturing machines, and provide real-time performance. The algorithm is demonstrated on six different machines spanning three technologies: laser fusion, binder jetting, and electron beam fusion. The authors show that the performance of the algorithm is superior to previous algorithms in terms of localization, accuracy, computation time, and generalizability. Gordon et al. (2020)\nocite{gordon2020defect} presented Defect Structure Process Maps (DSPMs) as a means of quantifying the role of porosity as an exemplary defect structure in powder bed printed materials. In 2021, Grasso et al. \nocite{grasso2021situ} conducted an updated review of research on in-situ sensing, measurement, and monitoring for metal powder bed fusion processes. This review included a classification of the various methods and a comparison of their performance. The authors noted the growth in the number of research studies in this field and the increasing use of sensors and data collection tools in industrial powder bed fusion systems. In addition, the review identified areas where additional technological advances are needed, as well as the types and sizes of defects that are detectable during production. Larsen et al. (2021) \nocite{larsen2022deep} investigated the use of deep representation learning to develop a physical model of the laser powder bed fusion process dynamics and created a predictive state space model that can be learned in a semi-supervised manner, resulting in a highly robust metric that correlated strongly with global material quality metrics. Yakout et al. (2021) \nocite{yakout2021situ} presented an in-situ monitoring setup for detecting spatter formation and delamination in Invar 36 (FeNi36) during the laser-based powder bed fusion process, and investigates the effects of various processing conditions on the formation of these phenomena. Harbig et al. (2022) presented a new methodology for evaluating the data from multiple process monitoring systems using sensor data fusion to improve the quality of defect detection in laser-based powder bed fusion of metals. The methodology is able to determine process anomalies and evaluate the suitability of specific process monitoring systems for defect detection, resulting in a defect detection of up to 92\% of melt track defects.\\
    Recent academic research on in situ detection in powder bed fusion (PBF) processes often employs machine learning and deep learning models to identify defects related to interactions between the recoater blade and the powder bed. Scime et al. (2018) utilized a convolutional neural network (CNN) for autonomous detection and classification of these spreading anomalies. Mahmoudi et al. (2019) proposed a method for layer-wise anomaly detection during laser PBF metal additive manufacturing. Elwarfalli et al. (2019) analyzed infrared images from a selective laser melting (SLM) machine using a computer-aided design (CAD) designed part with specific geometries on multiple layers for feature detection. Scime et al. (2020) presented a novel CNN architecture for pixel-wise localization of layer-wise powder bed imaging data. Zhang et al. (2020) proposed a method of hybrid CNNs for PBF process monitoring. Baumgartl et al. (2020) used thermographic off-axis imaging and deep learning-based neural networks to detect printing defects. Westphal et al. (2021) presented complex transfer learning methods for the automatic classification of powder bed defects in the SLM process using small datasets. Ertay et al. (2021) investigated the interaction between process parameters and the scan path on the occurrence of subsurface pores. Chen et al. (2022) utilized a two-stage CNN model for defect detection and segmentation.\\

    \textbf{Terminology 1 - Potential defects}\
    
    processing conditions on the formation of these phenomena. Harbig et al. (2022) presented a new methodology for evaluating the data from multiple process monitoring systems using sensor data fusion to improve the quality of defect detection in laser-based powder bed fusion of metals. The methodology is able to determine process anomalies and evaluate the suitability of specific process monitoring systems for defect detection, resulting in a defect detection of up to 92\% of melt track defects.\\
    As with any manufacturing process, there are potential defects that can occur during laser powder fusion bed processes. Some common defects include:
    \begin{description}
    \item[$\bullet$] Porosity: This refers to the presence of small voids or pores in the metal part. These can be caused by incomplete melting of the metal powders, or by gas bubbles trapped in the part.
    \item[$\bullet$ ] Incomplete fusion: This refers to areas where the metal powders have not completely fused together, resulting in a weak or porous part.
    \item[$\bullet$ ] Distortion: This refers to deformations or changes in the shape of the part caused by the heating and cooling process.
    \item[$\bullet$ ] Surface roughness: This refers to irregularities or variations in the surface finish of the part. This can be caused by factors such as the laser power, scan speed, and powder size.
    \item[$\bullet$ ] Residual stresses: This refers to stresses or strains that are present in the part after it has been printed. These can be caused by factors such as the printing temperature and part geometry.
    \end{description}
    Overall, these defects can affect the strength, accuracy, and overall quality of the metal parts produced using laser powder fusion bed processes. Proper process control and monitoring can help to minimize or prevent these defects.\\

    \textbf{Terminology 2 - The part and build geometry}\

    The part and build geometry, or the shape and dimensions of the object being printed and the way it is positioned in the build space, can influence the performance of laser powder fusion bed processes. In general, the part geometry can affect factors such as the printing speed, surface finish, and overall quality of the final part. The build geometry, on the other hand, can affect factors such as the amount of support material needed and the overall efficiency of the printing process.
    Here are some specific ways that part and build geometry can influence laser powder fusion bed processes:
    \begin{description}
    \item[$\bullet$ ] Part geometry: Complex or irregular part geometries can make it more difficult for the laser to melt and fuse the metal powders evenly, which can result in defects such as porosity or incomplete fusion. In addition, certain geometries may be more susceptible to distortion or residual stresses due to the heating and cooling process.
    \item[$\bullet$ ] Build geometry: The way a part is positioned in the build space can affect the amount of support material needed and the efficiency of the printing process. For example, printing a part with overhangs or cantilevers may require more support material, while printing multiple parts in a compact arrangement may reduce the overall printing time.
    \item[$\bullet$ ] Part orientation: The orientation of a part in the build space can affect the quality and surface finish of the final part. In general, printing with the part in a horizontal orientation may produce a smoother surface finish, while printing with the part in a vertical orientation may produce a rougher surface finish.
    \end{description}
    Overall, the part and build geometry can have a significant influence on the performance of laser powder fusion bed processes. Proper consideration of these factors can help to optimize the printing process and produce high-quality metal parts.